%%%%%%%%%%%%%%%%%%%%%%%%%%%%%%%%%%%%%%%%%
% Academic Title Page
% LaTeX Template
% Version 2.0 (17/7/17)
%
% This template was downloaded from:
% http://www.LaTeXTemplates.com
%
% Original author:
% WikiBooks (LaTeX - Title Creation) with modifications by:
% Vel (vel@latextemplates.com)
%
% License:
% CC BY-NC-SA 3.0 (http://creativecommons.org/licenses/by-nc-sa/3.0/)
% 
% Instructions for using this template:
% This title page is capable of being compiled as is. This is not useful for 
% including it in another document. To do this, you have two options: 
%
% 1) Copy/paste everything between \begin{document} and \end{document} 
% starting at \begin{titlepage} and paste this into another LaTeX file where you 
% want your title page.
% OR
% 2) Remove everything outside the \begin{titlepage} and \end{titlepage}, rename
% this file and move it to the same directory as the LaTeX file you wish to add it to. 
% Then add \input{./<new filename>.tex} to your LaTeX file where you want your
% title page.
%
%%%%%%%%%%%%%%%%%%%%%%%%%%%%%%%%%%%%%%%%%

%----------------------------------------------------------------------------------------
%  PACKAGES AND OTHER DOCUMENT CONFIGURATIONS
%----------------------------------------------------------------------------------------

\documentclass[11pt]{article}

\usepackage[utf8]{inputenc} 
\usepackage[T1]{fontenc} 

\usepackage{mathpazo} % Palatino font

\usepackage{graphicx} 
\usepackage[colorlinks,bookmarksopen,bookmarksnumbered]{hyperref}
\usepackage[spanish]{babel}
\usepackage{float}

\newcommand{\seclabel}[1]{\label{sec:#1}}
\newcommand{\figlabel}[1]{\label{fig:#1}}
\newcommand{\tablabel}[1]{\label{tab:#1}}
\newcommand{\chaplabel}[1]{\label{chap:#1}}
\newcommand{\figref}[1]{Figure~\ref{fig:#1}}
\newcommand{\secref}[1]{Section~\ref{sec:#1}}
\newcommand{\chapref}[1]{Chapter~\ref{chap:#1}}
\newcommand{\tabref}[1]{Table~\ref{fig:#1}}


\newcommand{\mynote}[2]{ 
\fbox{\bfseries\sffamily\scriptsize#1}
 \textcolor{red}{\small\textsf{\emph{#2}}}
\fbox{\bfseries\sffamily\scriptsize }}
\newcommand\mc[1]{\mynote{MC}{#1}}

\begin{document}

%----------------------------------------------------------------------------------------
%  TITLE PAGE
%----------------------------------------------------------------------------------------

\begin{titlepage} % Suppresses displaying the page number on the title page and the subsequent page counts as page 1
  \newcommand{\HRule}{\rule{\linewidth}{0.5mm}} % Defines a new command for horizontal lines, change thickness here
  
  \center % Centre everything on the page
  
  %------------------------------------------------
  %  Logo
  %------------------------------------------------

  \begin{figure}
    \centering
    \includegraphics[width=0.5\linewidth]{./images/logos/BRC_SinFondo.png}
  \end{figure}

  %------------------------------------------------
  %  Title
  %------------------------------------------------
  
  \HRule\\[0.4cm]
  
  {\Huge\bfseries Beauchef Robotic Challenge}\\[0.4cm] % Title of your document
  {\LARGE\bfseries Talleres: Reglas y Recomendaciones}
  \HRule\\[1.5cm]
  
  
  \vfill\vfill
  %------------------------------------------------
  %  Organization
  %------------------------------------------------

  {\Large\bfseries Organizan}

  \begin{figure}[h!]
    \centering
    \begin{minipage}{.24\textwidth}
      \centering
      \includegraphics[width=\linewidth]{./images/logos/ieee.png}
    \end{minipage}%
    \begin{minipage}{.24\textwidth}
      \centering
      \includegraphics[width=\linewidth]{./images/logos/comrob.png}
    \end{minipage}
    \begin{minipage}{.24\textwidth}
      \centering
      \includegraphics[width=\linewidth]{./images/logos/fablab.jpg}
    \end{minipage}
    \begin{minipage}{.24\textwidth}
      \centering
      \includegraphics[width=\linewidth]{./images/logos/ob.png}
    \end{minipage}
  \end{figure}

  
  \vfill\vfill\vfill % Position the date 3/4 down the remaining page
  %------------------------------------------------
  %  Date
  %------------------------------------------------
  
  
  {\large Version: \the\year} % Date, change the \today to a set date if you want to be precise
  
  %------------------------------------------------
  %  Logo
  %------------------------------------------------
  
  %\vfill\vfill
  %\includegraphics[width=0.2\textwidth]{placeholder.jpg}\\[1cm] % Include a department/university logo - this will require the graphicx package
   
  %----------------------------------------------------------------------------------------
  
  \vfill % Push the date up 1/4 of the remaining page
  
\end{titlepage}

%----------------------------------------------------------------------------------------

\section*{Acerca de}
Este es el libro oficial de reglas y recomendaciones de los talleres realizados previos a la competencia de robótica Beauchef Robotic Challenge (BRC).

Cualquier consulta, enviar un mail a: \href{mailto:beauchefrc@gmail.com}{beauchefrc@gmail.com}

% \section*{Agradecimientos}

% Queremos agradecer a los miembros de la Comunidad de Robótica, Fablab y IEEE UChile por adaptar las reglas y organizar la BRC.

% También queremos agradecer a las siguientes instituciones por su colaboración en la realización de BRC:

% \begin{itemize}
%    \item \textbf{Facultad de Ingeniería y Ciencias (FCFM)} de la Universidad de Chile por el apoyo entregado para la realización de esta competencia. 
%    \item  \textbf{Innovación y Robótica Estudiantil} de la Universidad Técnica Federico Santa María (UTFSM) por su invaluable apoyo en la creación de este libro y organización de BRC.
%  \end{itemize} 


\section*{Changelog}

\begin{itemize}
  \item \textbf{Julio 2019}
  \begin{itemize}
    \item Primera versión de este documento
  \end{itemize}
\end{itemize}

\pagebreak

\section{Introducción}

Junto con BRC, la organización coordina una serie de talleres y hackatones complementarias a la competencia, los cuales tienen por fin enseñar todas las herramientas para desarrollar un robot velocista básico. Estos talleres son \textbf{obligatorios para los equipos que reciben becas}, y se recomienda a todos los participantes asistir de todas formas, tanto miembros de la comunidad universitaria como externos a ella. 

Los talleres son realizados en la FCFM con un fuerte apoyo del FabLab UChile. Las personas que realicen los talleres tienen permitido hacer uso de las dependencias del FabLab para el diseño y construcción de sus robots de acuerdo a la normativa del laboratorio, y posterior al evento, quedar validados como usuarios del espacio \textbf{si es que pertenecen a la comunidad de la Universidad de Chile) 

\section{Talleres}

Como preparación para BRC, organizamos 5 talleres y 1 hackatón final.
Todos los eventos son abiertos a los participantes de la competencia.
Los talleres con fecha posterior al 14 de Agosto no podrán ser realizados en las dependencias del FabLab, sino que en una sala por definir de la FCFM. Los talleres corresponden a:

\paragraph*{Diseño CAD: Miércoles 31 de Julio: 12:00 - 13:30}

Operaciones básicas para diseñar en 3d utilizando el software Autodesk Fusion360. Estas habilidades permitirán a los equipos prototipar el chasis del robot y disponer de los componentes comerciales sobre él, así como tener las piezas para su fabricación digital. El uso de un computador personal es opcional. 

\paragraph*{Impresión 3D: Miércoles 5 de Agosto: 12:00 - 13:30}

Nociones básicas de impresión 3D FDM, configuración de impresión en el software MakerBotPrint y uso de impresora 3D Replicator2. Este taller permite a los usuarios acceso permanente a las impresoras del FabLab.

\paragraph*{Arduino y Electrónica Básica: Miércoles 14 de Agosto: 12:00 - 13:30}

Se enseña qué es un Arduino, lenguaje, funciones principales en un programa y algunos ejemplos, utilizando elementos de electrónica básica.
Es necesario que cada equipo traiga al menos un computador para poder realizar las experiencias prácticas.

\paragraph*{Prototipado Electrónico: Miércoles 21 de Agosto: 12:00 - 13:30}

Operaciones básicas en el Software Eagle para el diseño de placas electrónicas y uso de la máquina Roland Modela para la fabricación de estas. La placa electrónica no solamente se utiliza para sostener las componentes electrónicas firmes en el robot, sino que también suele ser el chasis del mismo, disminuyendo el peso total.

\paragraph*{Arduino y Control de Sistemas: Miércoles 28 de Agosto: 12:00 - 13:30}

Funciones avanzadas de Arduino, control de motores y control PID. En este taller se explicará la parte más importante del robot, que corresponde al control de los motores y los parámetros que rigen su funcionamiento, el PID.

*No es necesario asistir a los 4 primeros talleres si es que ya los realizaron. 

\paragraph*{Hackatón: Sábado 7 de Septiembre: Todo el día}
En la hackatón final guiamos a los competidores en la construcción de sus robots.
Miembros de la organización y diversos tutores apoyarán a los equipos todo el día en el FabLab, del cual se dispondrán sus máquinas.
La idea es que los equipos ya tengan parte del desarrollo avanzado y en esta instancia se resuelvan temas más complejos.

\section{Ganadores de Becas}

Para competidores becados, la participación en los talleres y hackatón final es \textbf{obligatoria}.
Si un equipo tiene problemas, se recomienda comunicarse lo antes posible con la organización.
Además, el uso de materiales del FabLab, como plástico para impresiones, es gratuito.

\section{Miembros de FCFM}

Se recomienda a los miembros no becados de la FCFM que se inscriban a los llamados de cada uno de los talleres en su momento. El costo del plástico de impresión 3d corresponde a 50 pesos el gramo.

\section{Miembros de la Universidad de Chile}

Se recomienda a los miembros de la universidad inscribirse en los talleres que deseen participar, el costo del plástico de impresión 3d corresponde a 100 pesos el gramo.

\section{Externos a la Universidad de Chile}
Si quieren participar en los talleres, los miembros ajenos a la comunidad de la Universidad de Chile deben inscribirse \textbf{obligatoriamente}. 
Como organización no aseguraremos el acceso a participantes que no se hayan inscritos.
El uso del FabLab se restringe exclusivamente al desarrollo del robot luego de realizados los talleres.
Los participantes deben inscribirse en la competencia BRC y mandar un correo con los nombres de los participantes que utilizarán las dependencias del FabLab y el equipo al que pertenecen.
Ante posibles dudas, contactar con la organización de BRC.

El costo de plástico de impresión es de 150 pesos el gramo.

\end{document}
